\documentclass{beamer}
\long\def\/*#1*/{}
\usepackage[czech]{babel}
\usepackage[utf8]{inputenc}
%\usepackage{multimedia}
\usepackage{animate,media9,movie15}
\title{Interaktivní
	 OpenGL demo}
\institute{{\large Fakulta informačních technologií\\ VUT}}
%\author{Patrik Chukir}
\author[Patrik Chukir]{Patrik Chukir\\{\small vedoucí práce: Tomáš Millét}}
\usetheme{Warsaw}

\date{{\small 30.1.2017}}

\begin{document}
	\begin{frame}
		<handout:0|beamer:0>
		\frametitle{Obsah}
		\tableofcontents
	\end{frame}
	\frame{\titlepage}
	

	\section{Zadání}
		\begin{frame}
		\frametitle{Zadání}
		\begin{itemize}[<+->]
			\item Prostudovat knihovnu OpenGl a její nástavby
			\item Navrhnout aplikaci umožňující zobrazení a interakci v 3D scéně
			\item Implementovat aplikaci
			\item Zhodnotit výsledek a navrhnout další postup
		\end{itemize}
		\end{frame}
	\section{Cíle práce}
	\begin{frame}
		\frametitle{Cíle práce}
		\begin{itemize}[<+->]
			\item Vytvořit scénu lukostřelecké střelnice
			\item Naprogramovat chování luku a šípů
			\item Zrealizovat změny počasí a osvětlovacích podmínek na scéně
			\item Naprogramovat vliv těchto změn na chování luku a šípů 
		\end{itemize}
	\end{frame}
	\section{Splněno}
		\begin{frame}
			\frametitle{Splněno}
			\begin{itemize}
				\item<1-> Vykreslování okna v OpenGL
					\begin{itemize}
						\item<2-> Zkompilování a nalinkování všech potřebných knihoven
						\item<2-> Vytvořit free kameru
						\item<2-> Zachytávání kláves
					\end{itemize}
				\item<3-> Načítání 3D modelů formátů wavefront a collada
				\item<4-> Načítání textur různých formátů pro tyto modely
				\item<5-> Jejich vykreslování
				\item<6-> Integrace \textit{Bullet physic engine}
					\begin{itemize}
						\item<6-> Gravitace
						\item<6-> Zachytávání kolizí a jejich základní ošetření
						\item<6-> Fyzikální chování pro kameru	
					\end{itemize}	
\end{itemize}
		\end{frame}
	\section{Plán na letní semestr}
		\begin{frame}
			\frametitle{Plán na letní semestr}	
			\begin{itemize}[<+->]
				\item Animace pohybu a deformace těles (\textit{Skeletal animation})
				\item Vyřešit práci s maticemi modelů, aby byly zachovány směry a místa objevování
				\item Zabodávání šípů
				\item Počasí
				\item Vliv počasí na let šípu a sílu luku	
			\end{itemize}
		\end{frame}
	\section{Poděkování}
	\begin{frame}	
		\begin{center}
				{\huge Děkuji za pozornost}
		\end{center}
	
	\end{frame}
\/*	\section{Ukázka}
		\begin{frame}
		%	\movie{poster text}{uk.avi}	
   \includemovie[poster,autoplay]{potee}{uk.avi}
		\end{frame}
	*/
\end{document}